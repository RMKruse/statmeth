% Options for packages loaded elsewhere
\PassOptionsToPackage{unicode}{hyperref}
\PassOptionsToPackage{hyphens}{url}
%
\documentclass[
]{article}
\usepackage{amsmath,amssymb}
\usepackage{lmodern}
\usepackage{iftex}
\ifPDFTeX
  \usepackage[T1]{fontenc}
  \usepackage[utf8]{inputenc}
  \usepackage{textcomp} % provide euro and other symbols
\else % if luatex or xetex
  \usepackage{unicode-math}
  \defaultfontfeatures{Scale=MatchLowercase}
  \defaultfontfeatures[\rmfamily]{Ligatures=TeX,Scale=1}
\fi
% Use upquote if available, for straight quotes in verbatim environments
\IfFileExists{upquote.sty}{\usepackage{upquote}}{}
\IfFileExists{microtype.sty}{% use microtype if available
  \usepackage[]{microtype}
  \UseMicrotypeSet[protrusion]{basicmath} % disable protrusion for tt fonts
}{}
\makeatletter
\@ifundefined{KOMAClassName}{% if non-KOMA class
  \IfFileExists{parskip.sty}{%
    \usepackage{parskip}
  }{% else
    \setlength{\parindent}{0pt}
    \setlength{\parskip}{6pt plus 2pt minus 1pt}}
}{% if KOMA class
  \KOMAoptions{parskip=half}}
\makeatother
\usepackage{xcolor}
\usepackage[margin=1in]{geometry}
\usepackage{color}
\usepackage{fancyvrb}
\newcommand{\VerbBar}{|}
\newcommand{\VERB}{\Verb[commandchars=\\\{\}]}
\DefineVerbatimEnvironment{Highlighting}{Verbatim}{commandchars=\\\{\}}
% Add ',fontsize=\small' for more characters per line
\usepackage{framed}
\definecolor{shadecolor}{RGB}{248,248,248}
\newenvironment{Shaded}{\begin{snugshade}}{\end{snugshade}}
\newcommand{\AlertTok}[1]{\textcolor[rgb]{0.94,0.16,0.16}{#1}}
\newcommand{\AnnotationTok}[1]{\textcolor[rgb]{0.56,0.35,0.01}{\textbf{\textit{#1}}}}
\newcommand{\AttributeTok}[1]{\textcolor[rgb]{0.77,0.63,0.00}{#1}}
\newcommand{\BaseNTok}[1]{\textcolor[rgb]{0.00,0.00,0.81}{#1}}
\newcommand{\BuiltInTok}[1]{#1}
\newcommand{\CharTok}[1]{\textcolor[rgb]{0.31,0.60,0.02}{#1}}
\newcommand{\CommentTok}[1]{\textcolor[rgb]{0.56,0.35,0.01}{\textit{#1}}}
\newcommand{\CommentVarTok}[1]{\textcolor[rgb]{0.56,0.35,0.01}{\textbf{\textit{#1}}}}
\newcommand{\ConstantTok}[1]{\textcolor[rgb]{0.00,0.00,0.00}{#1}}
\newcommand{\ControlFlowTok}[1]{\textcolor[rgb]{0.13,0.29,0.53}{\textbf{#1}}}
\newcommand{\DataTypeTok}[1]{\textcolor[rgb]{0.13,0.29,0.53}{#1}}
\newcommand{\DecValTok}[1]{\textcolor[rgb]{0.00,0.00,0.81}{#1}}
\newcommand{\DocumentationTok}[1]{\textcolor[rgb]{0.56,0.35,0.01}{\textbf{\textit{#1}}}}
\newcommand{\ErrorTok}[1]{\textcolor[rgb]{0.64,0.00,0.00}{\textbf{#1}}}
\newcommand{\ExtensionTok}[1]{#1}
\newcommand{\FloatTok}[1]{\textcolor[rgb]{0.00,0.00,0.81}{#1}}
\newcommand{\FunctionTok}[1]{\textcolor[rgb]{0.00,0.00,0.00}{#1}}
\newcommand{\ImportTok}[1]{#1}
\newcommand{\InformationTok}[1]{\textcolor[rgb]{0.56,0.35,0.01}{\textbf{\textit{#1}}}}
\newcommand{\KeywordTok}[1]{\textcolor[rgb]{0.13,0.29,0.53}{\textbf{#1}}}
\newcommand{\NormalTok}[1]{#1}
\newcommand{\OperatorTok}[1]{\textcolor[rgb]{0.81,0.36,0.00}{\textbf{#1}}}
\newcommand{\OtherTok}[1]{\textcolor[rgb]{0.56,0.35,0.01}{#1}}
\newcommand{\PreprocessorTok}[1]{\textcolor[rgb]{0.56,0.35,0.01}{\textit{#1}}}
\newcommand{\RegionMarkerTok}[1]{#1}
\newcommand{\SpecialCharTok}[1]{\textcolor[rgb]{0.00,0.00,0.00}{#1}}
\newcommand{\SpecialStringTok}[1]{\textcolor[rgb]{0.31,0.60,0.02}{#1}}
\newcommand{\StringTok}[1]{\textcolor[rgb]{0.31,0.60,0.02}{#1}}
\newcommand{\VariableTok}[1]{\textcolor[rgb]{0.00,0.00,0.00}{#1}}
\newcommand{\VerbatimStringTok}[1]{\textcolor[rgb]{0.31,0.60,0.02}{#1}}
\newcommand{\WarningTok}[1]{\textcolor[rgb]{0.56,0.35,0.01}{\textbf{\textit{#1}}}}
\usepackage{graphicx}
\makeatletter
\def\maxwidth{\ifdim\Gin@nat@width>\linewidth\linewidth\else\Gin@nat@width\fi}
\def\maxheight{\ifdim\Gin@nat@height>\textheight\textheight\else\Gin@nat@height\fi}
\makeatother
% Scale images if necessary, so that they will not overflow the page
% margins by default, and it is still possible to overwrite the defaults
% using explicit options in \includegraphics[width, height, ...]{}
\setkeys{Gin}{width=\maxwidth,height=\maxheight,keepaspectratio}
% Set default figure placement to htbp
\makeatletter
\def\fps@figure{htbp}
\makeatother
\setlength{\emergencystretch}{3em} % prevent overfull lines
\providecommand{\tightlist}{%
  \setlength{\itemsep}{0pt}\setlength{\parskip}{0pt}}
\setcounter{secnumdepth}{-\maxdimen} % remove section numbering
\ifLuaTeX
  \usepackage{selnolig}  % disable illegal ligatures
\fi
\IfFileExists{bookmark.sty}{\usepackage{bookmark}}{\usepackage{hyperref}}
\IfFileExists{xurl.sty}{\usepackage{xurl}}{} % add URL line breaks if available
\urlstyle{same} % disable monospaced font for URLs
\hypersetup{
  pdftitle={Lösung Zettel 0},
  hidelinks,
  pdfcreator={LaTeX via pandoc}}

\title{Lösung Zettel 0}
\author{}
\date{\vspace{-2.5em}}

\begin{document}
\maketitle

\hypertarget{luxf6sung-zu-uxfcbungszettel-0}{%
\subsection{Lösung zu Übungszettel
0}\label{luxf6sung-zu-uxfcbungszettel-0}}

\hypertarget{aufgabe-1}{%
\subsubsection{Aufgabe 1}\label{aufgabe-1}}

\hypertarget{a}{%
\paragraph{a)}\label{a}}

Was haben wir?

Die PRIOR Wahrscheinlichkeit, dass eine Person krank ist
\begin{equation}
  P(A) = 0.01 
\end{equation}

Die TRUE POSITIVE Rate bzw. Sensitivität

\begin{equation}
  P(B | A) = 0.9 
\end{equation}

Die TRUE NEGATIVE Rate bzw. Spezifizität \begin{equation}
  P(\bar{B} | \bar{A}) = 0.9 
\end{equation}

\begin{center}\rule{0.5\linewidth}{0.5pt}\end{center}

Was wollen wir? Posterior Wahrscheinlichkeit, dass eine Person krank
ist? \[
  P(A | B)
\]

\begin{center}\rule{0.5\linewidth}{0.5pt}\end{center}

Wie können wir es lösen? Wir schreiben auf, wie wir \(P(A | B)\)
berechnen.

\[
  P(A | B) = \dfrac{P(A \cap B)}{P(B)}
\]

Wir haben allerdings \(P(A \cap B)\) und \(P(B)\) nicht. Wie kann man
diese beide berechnen? \[
  P(A \cap B) = P(A) P(B | A) \text{ und } P(B) = P(A \cap B) + P(\bar A \cap B)
\] Wir haben all diese Werte, sodass wir nun alles ausrechnen können.

\begin{center}\rule{0.5\linewidth}{0.5pt}\end{center}

Rechenweg:

\begin{enumerate}
\def\labelenumi{\arabic{enumi}.}
\tightlist
\item
  Schritt: Schnittmengen berechnen
\end{enumerate}

\begin{align}

  P(A \cap B)       &= P(A)P(B | A) = 0.01 \times 0.9 = 0.009 \\
  P(\bar A \cap B)  &= P(\bar A)P(B | \bar A) = 0.099

\end{align}

\begin{enumerate}
\def\labelenumi{\arabic{enumi}.}
\setcounter{enumi}{1}
\tightlist
\item
  Schritt: Unconditional Wahrscheinlichkeit für positiven Test. \[
    P(B)=P(A \cap B) + P(\bar A \cap B)= 0.009 + 0.099 = 0.108
  \]
\item
  Schritt: \[
    P(A | B) = \dfrac{P(A \cap B)}{P(B)} = \dfrac{0.009}{0.108} = \dfrac{1}{12}
  \]
\end{enumerate}

\begin{center}\rule{0.5\linewidth}{0.5pt}\end{center}

\hypertarget{aufgabe-2}{%
\subsubsection{Aufgabe 2}\label{aufgabe-2}}

Berechne anhand einer Normalverteilung mit Mittelwert 500 und
Standardabweichung 100 mit der Funktion \texttt{pnorm} die
Wahrscheinlichkeit, Werte zwischen 200 und 800 aus dieser Verteilung zu
erhalten

\begin{Shaded}
\begin{Highlighting}[]
\FunctionTok{pnorm}\NormalTok{(}\DecValTok{800}\NormalTok{, }\AttributeTok{mean =} \DecValTok{500}\NormalTok{, }\AttributeTok{sd =} \DecValTok{100}\NormalTok{) }\SpecialCharTok{{-}} \FunctionTok{pnorm}\NormalTok{(}\DecValTok{200}\NormalTok{, }\AttributeTok{mean=}\DecValTok{500}\NormalTok{, }\AttributeTok{sd=} \DecValTok{100}\NormalTok{)}
\end{Highlighting}
\end{Shaded}

\begin{verbatim}
## [1] 0.9973002
\end{verbatim}

\begin{center}\rule{0.5\linewidth}{0.5pt}\end{center}

\hypertarget{aufgabe-3}{%
\subsubsection{Aufgabe 3}\label{aufgabe-3}}

Berechne die folgenden Wahrscheinlichkeiten. Wie hoch ist die
Wahrscheinlichkeit bei einer Normalverteilung mit Mittelwert 800 und
Standardabweichung 150, dass

\begin{itemize}
\tightlist
\item
  eine Punktzahl von 700 oder weniger
\item
  eine Punktzahl von 900 oder mehr
\item
  eine Punktzahl von 800 oder mehr
\end{itemize}

\begin{center}\rule{0.5\linewidth}{0.5pt}\end{center}

\textbf{Hinweis:} Die \texttt{pnorm()}-Funktion berechnet die
Wahrscheinlichkeit für einen Wert von \(-\infty\) bis \(x\).

\begin{Shaded}
\begin{Highlighting}[]
\FunctionTok{pnorm}\NormalTok{(}\DecValTok{700}\NormalTok{, }\AttributeTok{mean=}\DecValTok{800}\NormalTok{, }\AttributeTok{sd=}\DecValTok{150}\NormalTok{)}
\end{Highlighting}
\end{Shaded}

\begin{verbatim}
## [1] 0.2524925
\end{verbatim}

\textbf{Hinweis:} Wir wollen nun von \(x\) bis \(\infty\) berechnen,
hier für gibt es zwei Lösungswege: - Entweder berechnen wir die
Gegenwahrscheinlichkeit von \(-\infty\) bis \(x\) - Oder wir nutzen das
Argument \texttt{lower.tail=FALSE}, welche die Berechnung ``umdreht''.

\begin{Shaded}
\begin{Highlighting}[]
\CommentTok{\# 1. Weg}
\DecValTok{1} \SpecialCharTok{{-}} \FunctionTok{pnorm}\NormalTok{(}\DecValTok{900}\NormalTok{, }\AttributeTok{mean=}\DecValTok{800}\NormalTok{, }\AttributeTok{sd=}\DecValTok{150}\NormalTok{)}
\end{Highlighting}
\end{Shaded}

\begin{verbatim}
## [1] 0.2524925
\end{verbatim}

\begin{Shaded}
\begin{Highlighting}[]
\CommentTok{\# 2.Weg}
\FunctionTok{pnorm}\NormalTok{(}\DecValTok{900}\NormalTok{, }\AttributeTok{mean=}\DecValTok{800}\NormalTok{, }\AttributeTok{sd=}\DecValTok{150}\NormalTok{, }\AttributeTok{lower.tail =} \ConstantTok{FALSE}\NormalTok{)}
\end{Highlighting}
\end{Shaded}

\begin{verbatim}
## [1] 0.2524925
\end{verbatim}

\textbf{Hinweis:} Wir wollen nun von \(x\) bis \(\infty\) berechnen,
also entweder machen wir es wie in der Aufgabe davor, oder wir sehen,
dass wir hier einen Mittelwert Annahmen der gleich des
Verteilungsmittelwerts ist. Da wir eine symmetrische Verteilung haben,
berechnen wir also genau die Hälfe als 50\%.

\begin{Shaded}
\begin{Highlighting}[]
\CommentTok{\# 1. Weg}
\DecValTok{1} \SpecialCharTok{{-}} \FunctionTok{pnorm}\NormalTok{(}\DecValTok{800}\NormalTok{, }\AttributeTok{mean=}\DecValTok{800}\NormalTok{, }\AttributeTok{sd=}\DecValTok{150}\NormalTok{)}
\end{Highlighting}
\end{Shaded}

\begin{verbatim}
## [1] 0.5
\end{verbatim}

\begin{Shaded}
\begin{Highlighting}[]
\CommentTok{\# 2.Weg}
\FunctionTok{pnorm}\NormalTok{(}\DecValTok{800}\NormalTok{, }\AttributeTok{mean=}\DecValTok{800}\NormalTok{, }\AttributeTok{sd=}\DecValTok{150}\NormalTok{, }\AttributeTok{lower.tail =} \ConstantTok{FALSE}\NormalTok{)}
\end{Highlighting}
\end{Shaded}

\begin{verbatim}
## [1] 0.5
\end{verbatim}

\end{document}
